\documentclass[czech, kiv, he, sem, pdf, viewonly]{fasthesis}

\usepackage{pdfpages}

\usepackage{float}

\title{Karetní hra prší pro dva hráče realizovaná nad TCP}

\author{Jakub}{Vokoun}{}{}

\nobastardtitle \nocopyrightnotice

\begin{document} \frontpages[notm]

\tableofcontents

\chapter{Zadání}

Pro karetní hru prší, v konkrétních pravidlech: nedá se přebíjet sedmička ani
eso, měnič mění pouze na svojí barvu, naprogramujte server v jazyce C nebo C++
spustitelný běžnými nástroji na OS GNU/Linux. Vytvořte klienta v jiném jazyce
(zvolen Python) jako přenositelnou aplikaci, která podporuje GNU/Linux i MS
Windows.

Síťová hra pro více hráčů s architekturou 1:N (server - klienti) fungující nad
protokolem TCP, za použití standardních BSD socketů. Cílem semestrální práce je
fungující aplikace s dobrým návrhem protokolu a jeho implementací.

\chapter{Protokol}

Komunikace mezi serverem a klientem není řízená celou dobu z jedné strany.
Samotná hra je iniciovaná a řízená serverem, pohyb klienta po serveru je řízen
klientem.

Klient se může v jednu chvíli nacházet pouze v jednom stavu. Stavy jsou
následující:

\begin{itemize}

    \item unnamed

    \item lobby

    \item room

    \item game

\end{itemize}


\begin{figure}[H]

    \centering

    \includegraphics[width=0.5\textwidth]{img/protocol_flow.pdf}

    \caption{Průchod klienta serverem.}

    \label{fig:protocol_flow}

\end{figure}

% TODO: jake stavy ma room

Zprávy v protokolu mají proměnnou délku, v závislosti jak na typu zprávy, tak i
na přenášených datech. Protokol používá lidsky čitelná anglická slova psaná
velkými písmeny. Jednotlivé části zprávy jsou odděleny mezerou, zprávy jsou od
sebe odděleny znakem |. Protokol vypadá následovně:

\begin{center}

    \begin{longtable}{p{5cm} p{2cm} p{7cm}}

    \caption{Popis veškerých zpráv v protokolu. Pokud je zpráva variabilní, část z ní se opakuje v závislosti na něčem, je tato opakující se část uzavřena mezi symboly \# (které se v protokolu nevyužívají). Statické části zpráv jsou psány velkými písmeny, proměnné malými.}
    \label{tab:proto}\\

    \toprule[1.5pt] \textbf{zpráva} & \textbf{povolené stavy} & \textbf{popis} \\ \midrule
    \endfirsthead

    \multicolumn{3}{c}{\tablename{}~\thetable{} \textit{(pokračování z~předchozí
    stránky)}}\\ \midrule \textbf{zpráva} & \textbf{povolené stavy}& \textbf{popis} \\
    \midrule \endhead

    \midrule \multicolumn{3}{r}{\textit{(tabulka pokračuje na další stránce)}}\\
    \endfoot

    \bottomrule[1.5pt] \endlastfoot

    \multicolumn{3}{c}{\textbf{Zprávy iniciované klientem}}\\
    \midrule

    NAME string & unnamed & Klient si zvolí libovolné jméno.\\
    OK NAME & unnamed & Server potvrdí výběr jména.\\

    LIST\_ROOMS & lobby & Klient prosí server o seznam místností.\\
    ROOMS count \# id room-state \# & lobby & Server posílá seznam místností, každá má svůj číselný identifikátor (int) a stav (string).\\

    JOIN\_ROOM id & lobby & Klient žádá o připojení do místnosti.\\
    OK JOIN\_ROOM & lobby & Server potvrzuje připojení do místnosti.\\
    FAIL JOIN\_ROOM & lobby & Server nemohl přiřadit klienta do místnosti (byla plná, neexistovala).\\

    CREATE\_ROOM & lobby & Klient se pokouší vytvořit místnost.\\
    OK CREATE\_ROOM & lobby & Server vytvořil místnost a přiřadil do ní klienta.\\
    FAIL CREATE\_ROOM & lobby & Server nemohl vytvořit místnost (dosažen limit místností).\\

    ROOM\_INFO & room/game & Klient žádá podrobnější informace o místnosti, ve které se nachází.\\
    ROOM id room-state PLAYERS count \# name state \# & room/game & Server posílá informace o serveru.\\

    LEAVE\_ROOM & room/game & Klient opouští místnost.\\
    OK LEAVE\_ROOM & room/game (/lobby) & Server potvrzuje opuštění místnosti. Klientská aplikace tou dobou už může být v lobby, v závislosti na implementaci.\\


  \end{longtable}

\end{center}


Poznámky k protokolu:

\begin{itemize}

    \item Výběr jména nemůže neuspět, jelikož pokud si klient zvolí jméno
        stejné, jako již existující hráč, server předpokládá změnu koncového
        zařízení klienta a hráči nastaví nový FD, tedy přiřadí hráče novému
        socketu.

\end{itemize}


\backmatter \backpage \end{document}
