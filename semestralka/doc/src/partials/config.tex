\begin{center}

    \begin{longtable}{p{2cm} p{2cm} p{7cm}}

    \caption{Seznam všech konfiguračních proměnných pro server.}
    \label{tab:cfg}\\

    \toprule[1.5pt] \textbf{proměnná} & \textbf{výchozí} & \textbf{popis} \\ \midrule
    \endfirsthead

    \multicolumn{3}{c}{\tablename{}~\thetable{} \textit{(pokračování z~předchozí
    stránky)}}\\ \midrule \textbf{proměnná} & \textbf{výchozí} & \textbf{popis} \\
    \midrule \endhead

    \midrule \multicolumn{3}{r}{\textit{(tabulka pokračuje na další stránce)}}\\
    \endfoot

    \bottomrule[1.5pt] \endlastfoot

        IP string & 0.0.0.0 & Na jaké IP server poslouchá.\\
        PORT int & 3750 & Na jakém portu server naslouchá.\\
        MC int & 10 & Maximální počet klientů.\\
        MR int & 10 & Maximální počet místností.\\[1cm]

        \multicolumn{3}{c}{\textbf{nedoporučeno upravovat}}\\ \midrule
        EME int & 32 & Epoll max events.\\
        ET int & 500 & Epoll timeout [ms]. Za kolik ms přestane být epoll blokující.\\
        PT int & 2.000 & Frekvence posílání pingu v ms.\\
        ST int & 5.000 & Kolik ms bez pingu znamená, že je klient dočasně nedostupný.\\
        DT int & 180.000 & Kolik ms bez pingu, než je klient prohlášen za nedostupného.\\

  \end{longtable}

\end{center}
