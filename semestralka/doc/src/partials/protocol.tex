\begin{center}

  \begin{longtable} {cccl}

    \caption{Popis veškerých zpráv v protokolu. Pokud je zpráva variabilní, část z ní se opakuje v závislosti na něčem, je tato opakující se část uzavřena mezi symboly \# (které se v protokolu nevyužívají). Statické části zpráv jsou psány velkými písmeny, proměnné malými.}
    \label{tab:proto}\\

    \toprule[1.5pt] \textbf{server} & \textbf{klient} & \textbf{povolené stavy} & \textbf{popis} \\ \midrule
    \endfirsthead

    \multicolumn{4}{c}{\tablename{}~\thetable{} \textit{(pokračování z~předchozí
    stránky)}}\\ \midrule \textbf{server} & \textbf{klient} & \textbf{povolené stavy}& \textbf{popis} \\
    \midrule \endhead

    \midrule \multicolumn{4}{r}{\textit{(tabulka pokračuje na další stránce)}}\\
    \endfoot

    \bottomrule[1.5pt] \endlastfoot

    \multicolumn{4}{c}{\textbf{Zprávy iniciované klientem}}\\
    \midrule

    - & NAME string & unnamed & Klient si zvolí libovolné jméno.\\
    OK NAME & - & unnamed & Server potvrdí výběr jména.\\

    - & LIST\_ROOMS & lobby & Klient prosí server o seznam místností.\\
    ROOMS count \# id room-state \# & - & lobby & Server posílá seznam místností, každá má svůj číselný identifikátor (int) a stav (string).\\

    - & JOIN\_ROOM id & lobby & Klient žádá o připojení do místnosti.\\
    OK JOIN\_ROOM & - & lobby & Server potvrzuje připojení do místnosti.\\
    FAIL JOIN\_ROOM & - & lobby & Server nemohl přiřadit klienta do místnosti (byla plná, neexistovala).\\

    - & CREATE\_ROOM & lobby & Klient se pokouší vytvořit místnost.\\
    OK CREATE\_ROOM & - & lobby & Server vytvořil místnost a přiřadil do ní klienta.\\
    FAIL CREATE\_ROOM & - & lobby & Server nemohl vytvořit místnost (dosažen limit místností).\\

    - & ROOM\_INFO & room & Klient žádá podrobnější informace o místnosti, ve které se nachází.\\
    ROOM id room-state PLAYERS count \# name state \# & room/game & Server posílá informace o serveru.\\


  \end{longtable}

\end{center}
